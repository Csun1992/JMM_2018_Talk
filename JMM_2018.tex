%%%%%%%%%%%%%%%%%%%%%%%%%%%%%%%%%%%%%%%%%%%%%%%%%
%
% Date: February 3, 2017
% Location: CASPER @ Baylor University
% Length: 45 minutes
%
%%%%%%%%%%%%%%%%%%%%%%%%%%%%%%%%%%%%%%%%%%%%%%%%%

\documentclass{beamer}

\mode<presentation> {
  	\usetheme{CambridgeUS}
	\usecolortheme{baylorbears}
}

\definecolor{BaylorGreen}{RGB}{0,72,51}
\definecolor{BaylorGold}{RGB}{253,187,48}
\newcommand{\green}{\color{BaylorGreen}}
\newcommand{\gold}{\color{BaylorGold}}

%%%%If you want an INTERLOCKING BU in the top corner uncomment this%%%
\addtobeamertemplate{frametitle}{}{%
\begin{textblock*}{20mm}(0.97\textwidth,-0.75cm)
\includegraphics[scale=0.09]{BU_Interlock.eps}
\end{textblock*}}

%\addtobeamertemplate{frametitle}{}{%
%\begin{textblock*}{20mm}(0.96\textwidth,-0.79cm)
%\includegraphics[scale=0.05]{BUpaw.eps}
%\end{textblock*}}

%\addtobeamertemplate{frametitle}{}{%
%\begin{textblock*}{20mm}(0.84\textwidth,-0.63cm)
%\includegraphics[scale=0.15]{BLogo.eps}
%\end{textblock*}}
\usepackage{tikz}
\usetikzlibrary{positioning}
\usepackage{epstopdf}
\usepackage{textpos}
\usepackage{graphicx}
\usepackage{color}
\usepackage{epsfig}
\usepackage{amsthm}
\usepackage{amsmath}
\usepackage{amsfonts}
\usepackage{euscript}
\usepackage[english]{babel}
\usepackage[latin1]{inputenc}
\usepackage{times}
\usepackage{tikz}
\usepackage[T1]{fontenc}
\title[CASPER @ Baylor University]{A mixed  up-downwind
scheme for solving \\ a Heston stochastic volatility model \\ on variable grids}
\author[Chong Sun]{Chong Sun \\\vspace{1mm} \tiny\textit{jointly with Dr. Q. Sheng}}
\institute[]{Department of Mathematics\\ Center for Astrophysics, Space Physics and Engineering Research\\Baylor University\\~\\ \includegraphics[scale=0.5]{BULogo.eps}}
\date{{January 13th, 2018}}

\subject{The Talk}

%%%%%%%%%%%%%%%%%%%%%%%%%%%%%%%%%%%%%%%%%%%%%%%%%%%%%
%              My Commands
%%%%%%%%%%%%%%%%%%%%%%%%%%%%%%%%%%%%%%%%%%%%%%%%%%%%%

\newtheorem{thm}{Theorem}
\newcommand{\ben}{\begin{enumerate}}
\newcommand{\een}{\end{enumerate}}
\newcommand{\bit}{\begin{itemize}}
\newcommand{\eit}{\end{itemize}}

\newcommand{\mypm}{\mathbin{\tikz [x=1.4ex,y=1.4ex,line width=.1ex] \draw (0.0,0) -- (1.0,0) (0.5,0.08) -- (0.5,0.92) (0.0,0.5) -- (1.0,0.5);}}%

\newcommand{\bigo}[1]{\left( #1 \right) }
\newcommand{\no}{\noindent}
\newcommand{\norm}[1]{\l|\l| #1 \r|\r|}

\newcommand{\vb}[1]{\textbf{#1}}
\newcommand{\bb}[1]{\begin{equation}\label{#1}}
\newcommand{\ee}{\end{equation}}
\newcommand{\bbb}{\begin{eqnarray}}
\newcommand{\eee}{\end{eqnarray}}
\newcommand{\bbbb}{\begin{eqnarray*}}
\newcommand{\eeee}{\end{eqnarray*}}
\newcommand{\nnn}{\nonumber}
\everymath{\displaystyle}
% Appendix Commands
\def \func {\mathrm}\def \limfunc {\mathrm}
\def\l{\left}
\def\r{\right}
\def\TT{\mathbb{T}}
\def\RR{\mathbb{R}}
\def\CC{\mathbb{C}}
\def\AA{{\cal{A}}}
\def\DD{{\cal{D}}}
\def\OO{{\cal{O}}}
\def\R#1{$(\ref{#1})$}
\newcommand{\tT}{\intercal}
\definecolor{green1}{rgb}{0.1,0.5,0.0}
\definecolor{blue1}{RGB}{51,78,125}
\definecolor{brown}{RGB}{220,90,49}
\definecolor{red1}{rgb}{0.8,0.1,0.0}
\definecolor{orange}{rgb}{1,0.5,0}

\newcommand{\darkblue}{\color{blue1}}
\newcommand{\darkred}{\color{red1}}
\newcommand{\orange}{\color{orange}}
\newcommand{\blue}{\color{blue}}
\newcommand{\red}{\color{red}}

\hfuzz=0.5pt
\vfuzz=0.5pt


%%%%%%%%%%%%%%%%%%%%%%%%%%%%%%%%%%%%%%%%%%%%%%%%%%%%%

\begin{document}

\begin{frame}
\titlepage
\end{frame}


\section{I.Heston Stochastic Differential Equation}

 \begin{frame}
\frametitle{Heston Stochastic Volatility Model}
Heston proposed that stock prices $S(t)$ and the associated volatility $y(t)$ follow a Brownian motion,
\bbbb
\text{d}S(t) &=& \mu S(t) \text{d}t+S(t)\sqrt{y(t)}\text{d}B(t)\\
\text{d}y(t) &=& \kappa [\eta -y(t)]\text{d}t+\sigma\sqrt{y(t)}\text{d}\tilde{B}(t)\\
\text{d}B(t)\text{d}\tilde{B}(t) &=& \rho \text{d}t
\eeee

\end{frame}

%%%%%%%%%%%%%%%%%%%%%%%%%%%%%%%%%%%%%%%%%%%%%%%%%%%%%

\section{II. Finite Difference Scheme}

\begin{frame}
\frametitle{Heston Partial Differential Equation}

\bbbb
V_\tau = \frac{1}{2}yV_{xx}+\rho \sigma yV_{xy}+\frac{1}{2}\sigma^2 yV_{yy}-\l(\frac{1}{2}y-r\r)V_x+\kappa (\eta -y)V_y,
\eeee	
										
\pause

\bbbb
V(x,y,0) &=& \max\l(1-e^{x},0\r),~~~ x\in \mathbb{R},~~  y\in \mathbb{R}^+,\\
\pause
\lim_{x\rightarrow -\infty}V(x,y,\tau) &=& 1,~~~ y\in\mathbb{R}^+,~~ \tau\in\mathbb{R}^+,\\
\lim_{x\rightarrow \infty}V(x,y,\tau) &=& 0,~~~ y\in\mathbb{R}^+,~~ \tau\in\mathbb{R}^+,\\
V(x,0,\tau) &=& \max\l(1-e^{x},0\r),~~~ x\in\mathbb{R},~~ \tau\in\mathbb{R}^+.\\
\pause
\lim_{y\rightarrow \infty}V_y(x,y,\tau) &=& 0,~~~ x\in\mathbb{R},~~ \tau\in\mathbb{R}^+,
\eeee

\end{frame}

%%%%

\begin{frame}
\frametitle{Traditional Approaches and Limitation}
Approaches:
\bit
\item Central difference approximation
\pause
\item von Neumann method for stability analysis
\eit
\pause
\vspace{3mm}
Limitation:
\bit
\item von Neumann analysis can only be applied to Cauchy problems or periodic boundary conditions
\eit

\end{frame}

%%%%

\begin{frame}
\frametitle{Our Approach--Mixed Derivative}
Mixed Derivative Term:
\bit
\item Positive coefficient: 
\bbbb
V_{xy}(x_m,y_n,\tau) \approx \frac{1}{2}(\Delta _{x,-}\Delta_{y,-}+\Delta_{x,+}\Delta_{y,+}) V(x_m,y_n,\tau).
\eeee

\pause

\item Negative coefficient:
\bbbb
V_{xy}(x_m,y_n,\tau) \approx \frac{1}{2}(\Delta _{x,+}\Delta_{y,-}+\Delta_{x,-}\Delta_{y,+}) V(x_m,y_n,\tau).
\eeee
\eit

\end{frame}

%%%%

\begin{frame}
\frametitle{Our Approach--Advection Terms}
\bit
\item Positive coefficient: Forward Difference Approximation
\vspace{2mm}
\item Negative coefficient: Backward Difference Approximation
\eit
\end{frame}

%%%%

\begin{frame}
\frametitle{Semi-Discretised System}
Semi-discretized system:
\bbbb
\label{semi}
\textbf{u}'(\tau) = \textbf{M}\textbf{u}(\tau)+\textbf{f}(\tau),
\eeee

The solution is 
\bbbb
\mathbf{u}(\tau) = e^{\tau \mathbf{M}}\mathbf{u}(0)-\int_0^{\tau} e^{(\tau -s)\textbf{M}}\textbf{f}(s)\text{d}s.
\eeee

\end{frame}

%%%%

\begin{frame}
\frametitle{Definition of Stability of Semi-Discretised Systems}

\begin{definition}[Stability of Semi-Discretised Systems]
The semi-discretised system is stable if for every $\tau^*>0$, there exists a constant $c(\tau^*)>0$ such that
\bbb
\|e^{\tau \mathbf{M}}\| \leq c(\tau^*),\quad \tau\in[0,\tau^*].
\eee
where $\|\cdot\|$ is an appropriate matrix norm.

\end{definition}


\end{frame}

%%%%

\begin{frame}
\frametitle{Gerschgorin's Circle and Exponential Behavior Theorems}
\begin{theorem}[Gerschgorin's Circle Theorem/Brauer's Theorem]

Let $M_s$ be the sum of the moduli of the elements along the $s$th row of matrix $\textbf{M}$ excluding the diagonal element $m_{ss}$. Then each eigenvalue of $\textbf{M}$ lies inside or on the boundary of at least one of the circles $|\lambda -m_{ss}|=M_s$. 
\end{theorem}

\pause

\begin{theorem}[Exponential Behavior]
$e^{t\textbf{A}}$ tends to 0 in certain norm hence in all norms, as $t$ tends to $+\infty$ , if and only if all the eigenvalues of $\textbf{A}$ have strictly negative real parts.
\end{theorem}
\pause
\begin{theorem}
For $\rho\in[-1,1]$, the semi-discretised system is stable.
\end{theorem}
\end{frame}

%%%%

\section{III. Numerical Experiments}
\begin{frame}
\frametitle{Domain Truncation}

\bbbb
V_\tau = \frac{1}{2}yV_{xx}+\rho \sigma yV_{xy}+\frac{1}{2}\sigma^2 yV_{yy}-\l(\frac{1}{2}y-r\r)V_x+\kappa (\eta -y)V_y,
\eeee	
										

\bbbb
V(x,y,0) &=& \max\l(1-e^{x},0\r),~~~ x\in [-X,~X],~~ y\in [0,~Y],\\
V(-X,y,\tau) &=& 1,~~~y\in [0,~Y], ~~\tau\in\mathbb{R}^+,\\
V(X,y,\tau) &=& 0,~~~y\in [0,~Y], ~~\tau\in\mathbb{R}^+,\\
V(x,0,\tau) &=& \max\l(1-e^{x},0\r),~~~ x\in [-X,~X], ~~\tau\in\mathbb{R}^+.\\
V_y(x,Y,\tau) &=& 0,~~~ x\in [-X,~X],~~ \tau\in\mathbb{R}^+,
\eeee

\end{frame}

%%%%

\begin{frame}
\frametitle{Solution Surface}
\begin{figure}
\begin{center}

{\epsfig{file=c123017surf1.eps,width=4in,height=2in}}
% {\epsfig{file=solution.eps,width=4in,height=2in}}

\caption{Price of an European put option}\label{figure:1}
\end{center}
\end{figure}
\end{frame}

\begin{frame}
\frametitle{Convergence Surface}
\begin{figure}
\begin{center}

{\epsfig{file=c123017conv1.eps,width=4in,height=2in}}
% {\epsfig{file=c122617a.eps,width=4in,height=2in}}

\caption{Rate of convergence $\rho^h_{PW}$ surface at  $T=0.5.$ The figure indicates approximately an order one rate of convergence.}\label{figure:5}
\end{center}
\end{figure}
\end{frame}
%%%%%%%%%%%%%%%%%%%%%%%%%%%%%%%%%%%%%%%%%%%%%%%%%%%%%%%

\section{IV. Future Work :}
\begin{frame}
\frametitle{Future Work}
\bit
\item Exponential Splitting 

\item Adaptive Grids

\item Higher-Order Schemes

\item Free Boundary Value Problems

\eit

\end{frame}
\section{V. The End}
\begin{frame}
\frametitle{}
\begin{center}
    {\green\huge Thank You}
\end{center}
\end{frame}

\end{document}
